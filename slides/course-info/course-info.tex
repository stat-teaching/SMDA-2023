% Options for packages loaded elsewhere
\PassOptionsToPackage{unicode}{hyperref}
\PassOptionsToPackage{hyphens}{url}
%
\documentclass[
  ignorenonframetext,
]{beamer}
\usepackage{pgfpages}
\setbeamertemplate{caption}[numbered]
\setbeamertemplate{caption label separator}{: }
\setbeamercolor{caption name}{fg=normal text.fg}
\beamertemplatenavigationsymbolsempty
% Prevent slide breaks in the middle of a paragraph
\widowpenalties 1 10000
\raggedbottom
\setbeamertemplate{part page}{
  \centering
  \begin{beamercolorbox}[sep=16pt,center]{part title}
    \usebeamerfont{part title}\insertpart\par
  \end{beamercolorbox}
}
\setbeamertemplate{section page}{
  \centering
  \begin{beamercolorbox}[sep=12pt,center]{part title}
    \usebeamerfont{section title}\insertsection\par
  \end{beamercolorbox}
}
\setbeamertemplate{subsection page}{
  \centering
  \begin{beamercolorbox}[sep=8pt,center]{part title}
    \usebeamerfont{subsection title}\insertsubsection\par
  \end{beamercolorbox}
}
\AtBeginPart{
  \frame{\partpage}
}
\AtBeginSection{
  \ifbibliography
  \else
    \frame{\sectionpage}
  \fi
}
\AtBeginSubsection{
  \frame{\subsectionpage}
}
\usepackage{amsmath,amssymb}
\usepackage{lmodern}
\usepackage{iftex}
\ifPDFTeX
  \usepackage[T1]{fontenc}
  \usepackage[utf8]{inputenc}
  \usepackage{textcomp} % provide euro and other symbols
\else % if luatex or xetex
  \usepackage{unicode-math}
  \defaultfontfeatures{Scale=MatchLowercase}
  \defaultfontfeatures[\rmfamily]{Ligatures=TeX,Scale=1}
\fi
\usetheme[]{SimpleDarkBlue}
% Use upquote if available, for straight quotes in verbatim environments
\IfFileExists{upquote.sty}{\usepackage{upquote}}{}
\IfFileExists{microtype.sty}{% use microtype if available
  \usepackage[]{microtype}
  \UseMicrotypeSet[protrusion]{basicmath} % disable protrusion for tt fonts
}{}
\makeatletter
\@ifundefined{KOMAClassName}{% if non-KOMA class
  \IfFileExists{parskip.sty}{%
    \usepackage{parskip}
  }{% else
    \setlength{\parindent}{0pt}
    \setlength{\parskip}{6pt plus 2pt minus 1pt}}
}{% if KOMA class
  \KOMAoptions{parskip=half}}
\makeatother
\usepackage{xcolor}
\newif\ifbibliography
\setlength{\emergencystretch}{3em} % prevent overfull lines
\providecommand{\tightlist}{%
  \setlength{\itemsep}{0pt}\setlength{\parskip}{0pt}}
\setcounter{secnumdepth}{-\maxdimen} % remove section numbering
\usepackage{academicons}
\usepackage{fontawesome}

\newif\ifshowtoc
\showtoctrue% toggles to show the toc

% % highlight current section
% \AtBeginSection{%
% \ifshowtoc
% \begin{frame}
%     \tableofcontents[currentsection, hideallsubsections, subsectionstyle=show/show/hide]
% \end{frame}
% \fi
% }

\setbeamercolor{section name}{fg=white}

% reduce font size footnotes
\renewcommand{\footnotesize}{\scriptsize} 

% command for footnotes without number
\newcommand\addnote[1]{%
  \begingroup
  \renewcommand\thefootnote{}\footnote{#1}%
  \addtocounter{footnote}{-1}%
  \endgroup
}

% multi columns

\newenvironment{cols}[1][]{}{}

\newenvironment{col}[1]{\begin{minipage}{#1}\ignorespaces}{%
\end{minipage}
\ifhmode\unskip\fi
\aftergroup\useignorespacesandallpars}

\def\useignorespacesandallpars#1\ignorespaces\fi{%
#1\fi\ignorespacesandallpars}

\makeatletter
\def\ignorespacesandallpars{%
  \@ifnextchar\par
    {\expandafter\ignorespacesandallpars\@gobble}%
    {}%
}
\makeatother

\setbeamerfont{section in toc}{size=\scriptsize}

\usepackage{booktabs}
\usepackage{xcolor}
\usepackage{booktabs}
\usepackage{longtable}
\usepackage{array}
\usepackage{multirow}
\usepackage{wrapfig}
\usepackage{float}
\usepackage{colortbl}
\usepackage{pdflscape}
\usepackage{tabu}
\usepackage{threeparttable}
\usepackage{csquotes}
\usepackage{makecell}
\ifLuaTeX
  \usepackage{selnolig}  % disable illegal ligatures
\fi
\IfFileExists{bookmark.sty}{\usepackage{bookmark}}{\usepackage{hyperref}}
\IfFileExists{xurl.sty}{\usepackage{xurl}}{} % add URL line breaks if available
\urlstyle{same} % disable monospaced font for URLs
\hypersetup{
  pdftitle={Introduction to Generalized Linear Models},
  pdfauthor={Filippo Gambarota},
  hidelinks,
  pdfcreator={LaTeX via pandoc}}

\title{Introduction to Generalized Linear Models}
\subtitle{General course information}
\author{Filippo Gambarota}
\date{2022/2023}
\institute{University of Padova}

\begin{document}
\frame{\titlepage}

\begin{frame}{About me}
\protect\hypertarget{about-me}{}
\begin{itemize}
\tightlist
\item
  I'm a post-doc researcher at the Department of Developmental
  Psychology and Socialization
\item
  I did a PhD in Experimental Psychology studying the unconscious
  working memory processing
\item
  I work with Professor Gianmarco Altoè on data analysis in Psychology,
  especially meta-analysis
\end{itemize}
\end{frame}

\begin{frame}{Office hours}
\protect\hypertarget{office-hours}{}
I do not have official office hours but we can schedule an appointment.
you can write me at
\textbf{\href{mailto::filippo.gambarota@unipd.it}{filippo.gambarota@unipd.it}}:

\begin{itemize}
\tightlist
\item
  my office is the 027, first floor Psico1 building (pink building)
\item
  we can also schedule on Zoom
\end{itemize}
\end{frame}

\begin{frame}{Materials}
\protect\hypertarget{materials}{}
The slides will be structured intermixing:

\begin{itemize}
\tightlist
\item
  R code
\item
  Theory and Formulas (not a lot :) )
\item
  Plots (a lot!)
\item
  Examples and exercises
\end{itemize}
\end{frame}

\begin{frame}{Materials}
\protect\hypertarget{materials-1}{}
\begin{itemize}
\tightlist
\item
  slides with the \texttt{\textcolor{red}{\#extra}} tag are very
  specific but useful topics that will be eventually covered but are not
  part of the core course/exam
\end{itemize}
\end{frame}

\begin{frame}[fragile]{R code}
\protect\hypertarget{r-code}{}
I mainly use R for my daily work but a deep understanding of R is not
necessary.

Slides are created with R Markdown (\texttt{.rmd} files) and distributed
in \texttt{pdf} and all source scripts are available.

I wrote several custom functions that are used in the slide and maybe
during the exercizes.
\end{frame}

\begin{frame}{Theory and Formulas}
\protect\hypertarget{theory-and-formulas}{}
I tried to reduce the amount of formulas. I prefer to make practical
examples and showing the R code.

The probability of making an error or typo is close to 1 (Shepard,
2023), if you find something strange raise the hand or write me an email
:)

\begin{center}\includegraphics[width=0.3\linewidth]{img/meme-math} \end{center}

\addnote{Source: https://www.splashlearn.com/blog/funny-school-memes-every-student-will-love/}
\end{frame}

\begin{frame}{Examples and exercises}
\protect\hypertarget{examples-and-exercises}{}
I tried to make practical examples whenever possible and relevant.
Furthermore we will see some exercises and case studies.
\end{frame}

\begin{frame}{Rules of the game}
\protect\hypertarget{rules-of-the-game}{}
\begin{enumerate}
\tightlist
\item
  \textbf{Participate!} If you have questions, doubts, comments, etc.
  please ask
\item
  If something is \textbf{not clear or is discordant} with other
  information from previous courses, tell me.
\item
  \textbf{Participate!}
\item
  Try do to \textbf{exercises and case studies}
\item
  \textbf{Participate!}
\item
  If you can, \textbf{bring your laptop with R}
\end{enumerate}
\end{frame}

\begin{frame}{Final note}
\protect\hypertarget{final-note}{}
This is my first teaching experience with this course. Suggestions and
critique are welcome.
\end{frame}

\begin{frame}[plain]
\begin{center}
  \faEnvelope \hspace{0.2cm} filippo.gambarota@unipd.it\\
  \vspace{0.5cm}
  \faGithub \hspace{0.2cm} \href{github.com/filippogambarota}{github.com/filippogambarota}
\end{center}
\end{frame}

\end{document}
